% the acknowledgments section

% \newthought{Lorem ipsum dolor sit amet}, consectetuer adipiscing elit. Morbi commodo, ipsum sed pharetra gravida, orci magna rhoncus neque, id pulvinar odio lorem non turpis. Nullam sit amet enim. Suspendisse id velit vitae ligula volutpat condimentum. Aliquam erat volutpat. Sed quis velit. Nulla facilisi. Nulla libero. Vivamus pharetra posuere sapien. Nam consectetuer. Sed aliquam, nunc eget euismod ullamcorper, lectus nunc ullamcorper orci, fermentum bibendum enim nibh eget ipsum. Donec porttitor ligula eu dolor. Maecenas vitae nulla consequat libero cursus venenatis. Nam magna enim, accumsan eu, blandit sed, blandit a, eros.

Al Patrick, per la paciència dels primers anys i la confiança dels últims. Mai m’hagués imaginat que podria aprendre tant durant aquest temps i que acabaria sentint que el lab és realment casa meva. I també al Xavi, per fer-ho tot tan fàcil i estar sempre disposat a donar un cop de mà. Han sigut uns mentors excepcionals.

El grup em va rebre amb els braços oberts i amb la voluntat de fer ben fàcil i agradable la meva estada. Començant pel Miquel, un mentor fantàstic, una ment brillant. Del Martino n’he après moltes coses durant tot aquest temps i, curiosament, les més importants no són de ciència. És una persona increïble. L’Aleix va ser sempre un referent per a mi: començar el doctorat amb ell al grup ho va fer tot molt més senzill. I, és clar, l’Adrià. Em va rebre com un germà i em va tractar com a tal durant els anys posteriors: aquesta tesi no hagués arribat a bon port sense el seu suport i les seves (nostres) tonteries. I, òbviament, a la resta de persones que han passat pel grup durant aquest temps: Lídia, Nico, Pau B, Martina, Paula, Roger, Ángel, Angelo, Eva, Nils, Marta, Pablo, Alexandre, Teo, Aksel, Miguel, Manel, Pau A i tants altres. Ha sigut una experiència fantàstica. 

A tots els companys de l’IRB amb qui he pogut compartir cool-offs, partits de futbol i grups d’interès. Especialment a l’Amanda, Aish, Berta, Elena, KG, Valentín, Luca, Alba, Levi i Hania. També als membres del grup del Xavi amb qui he tingut la sort de coincidir més sovint: Marina, Álvaro, Andrea i, evidentment, al Jordi i al Carles. Gràcies per acollir-me sempre com ho heu fet. També a tots aquells amb qui he col·laborat durant aquests anys, especialment a la Maria i a la Natalie. 

A la família, especialment als quatre avis. Avi, sé que continuaries retallant notícies del diari per a mi. I també als cosins, perquè per a mi sempre han estat com germans. Als amics de l’escola, als de la uni i als més recents. Ja sabeu qui sou. I, evidentment, a la Jesús, a l’Isaac, al Ramon i al Facund, per ser el meu refugi en els moments més durs.

Als membres actuals del grup. Al Carles, per portar les millors catànies i explicar anècdotes del segle passat. A la Yassmin, pel coffee-coffee i per ensenyar-me que els patrons irracionals poden tenir, a vegades, un cert component terapèutic. Al Bryan, el meu panchito de confiança, el creador oficial de memes i el centre d’atenció diari a la feina. És el nostre pallasso i li encanta. Al Guillem, per posar una mica de seny en aquest lab de bojos i intentar donar un xic d’emoció a la partida del Biwenger. Al Jesús i a la Judit, que tot i arribar fa pocs mesos al lab és ben bé com si hi portessin anys. És difícil no passar-s’ho bé al seu costat. I amb l’Aurora, el futur (i l’humor) del grup està en molt bones mans.

Sovint diuen a La Sotana que la vida és un tamboret de tres potes: la salut, l’amor i els diners. I, sabeu què? S’equivoquen. La quarta pota és l’amistat. 

Al Dylan, per ensenyar-me a prendre’m la vida amb més alegria i menys preocupacions, per molestar-me (en el sentit anglès del terme) cada dia i per, en definitiva, convertir-se en un dels meus millors amics (iate, el turras). A l’Elena, per ser com una germana petita, ensenyar-me a confiar en mi mateix, escoltar-me i aconsellar-me sempre. I a la Gema, l’altra germana, per estar al meu costat tots aquests anys (literalment), ensenyar-me a ser perseverant i aportar-me suport i tranquilitat. No hi ha concepte que hagi après, congrés al que hagi assistit o paper que hagi publicat comparable a guanyar amics per tota la vida. 

Finalment i, per tant, l’agraïment més important: als meus pares. Per ser-hi sempre, incondicionalment. Sospito que sóc el seu fill preferit.

 