
La representació d'entitats bioquímiques, com ara molècules petites i cavitats farmacològicament aprofitables de proteïnes, mitjançant descriptors numèrics facilita l'ús de tecnologies innovadores inspirades en la intel·ligència artificial en el camp del descobriment de fàrmacs. De fet, els vectors numèrics proporcionen representacions simplificades però potents, que permeten comparacions eficients i una millor interpretabilitat. En el cas de les molècules petites, la immensitat de l'espai químic farmacològicament aprofitable, essencialment infinit, roman en gran part inexplorat, i la seva exploració amb descriptors ofereix accés a una gran quantitat de compostos potencialment bioactius. Pel que fa a les cavitats de les proteïnes -els objectius principals dels fàrmacs basats en molècules petites-, la seva representació numèrica accelera la predicció de les funcions de les proteïnes i facilita l'avaluació de la polifarmacologia dels compostos. 

En aquesta tesi, pretenem ampliar el concepte de descriptors de molècules petites i cavitats susceptibles de ser objectius de fàrmacs, abordant les limitacions principals dels mètodes existents i dissenyant estratègies noves per millorar la seva capacitat de captar característiques estructurals, químiques i funcionals relevants. Concretament, hem (i) utilitzat signatures químiques per agrupar i comparar biblioteques de compostos, (ii) il·lustrat l'adveniment de les signatures de bioactivitat per permetre una exploració biològicament rellevant de l'espai químic i (iii) implementat una estratègia computacional predefinida d'integració de dades per crear noves signatures de bioactivitat. A més a més, també hem (iv) avaluat la relació entre l'estereoisomeria i la bioactivitat a una escala sense precedents, mentre dissenyavem nous signaturitzadors de compostos sensibles a l'esereoisomeria. Basant-nos en aquests avenços en la caracterització de molècules petites, hem dirigit el nostre enfocament als objectius proteics, (v) dissenyant i avaluant àmpliament una nova metodologia per generar descriptors numèrics de cavitats basats en les seves estructures tri-dimensionals, anomenada PocketVec. Aquesta nova estratègia, basada en el principi de la quimiogenòmica que estableix que cavitats similar s'uneixen a compostos similars, ens ha permès (vi) carateritzar i explorar tot el proteoma humà des de la perspectiva de les cavitats, identificant cavitats similars farmacològicament aprofitables en proteïnes sense relació aparent. En conjunt, aquesta tesi representa una modesta contribució als camps en creixement constant de la quimioinformatica i la bioinformàtica, enmarcada dins el procés més ampli i llarg del descobriment de fàrmacs.