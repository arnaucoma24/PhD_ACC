% the abstract

The representation of biochemical entities, such as small molecules and protein druggable pockets, with numerical descriptors facilitates the use of novel AI-inspired technologies in drug discovery. Indeed, numerical vectors provide simplified yet powerful representations, amenable to efficient comparisons and improved interpretability. For small molecules, the vastness of the essentially infinite drug-like chemical space remains largely uncharted, and its exploration with descriptors offers a wealth of potentially bioactive compounds. For protein pockets -the primary targets of small molecule drugs- numerical representation expedites the prediction of protein function as well as supporting the evaluation of compound polypharmacology.  

In this thesis, we aim at further extending the concept of small molecule and druggable pocket descriptors, addressing the main limitations of existing methods and designing novel approaches to enhance their ability to capture relevant structural, chemical and functional features. More specifically, we have (i) leveraged chemical signatures to cluster and compare compound libraries, (ii) illustrated the advent of bioactivity signatures to enable the biologically relevant exploration of the chemical space, and (iii) implemented a predefined data integration framework to create novel bioactivity signatures. Moreover, we have also (iv) assessed the relationship between stereoisomerism and bioactivity at an unprecedented scale while designing stereochemically-aware compound signaturizers. Building on these advancements in small molecule characterization, we have expanded our focus to protein targets by (v) designing and extensively benchmarking a new methodology to generate structure-based numerical pocket descriptors, named PoketVec. This novel approach, founded on the chemogenomics principle stating that similar pockets will bind similar compounds, has enabled us to (vi) characterize, and explore, the complete human proteome from the pocket perspective, identifying similar druggable pockets in otherwise unrelated protein targets. Overall, this thesis represents a modest contribution to the ever-growing fields of cheminformatics and bioinformatics, encompassed within the wider and long-term process of drug discovery. 