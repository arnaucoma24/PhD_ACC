\subsection{Results and discussion}
\phantomsection

\subsubsection{Reducing the size of chemical libraries}

To create a new version of the in-house (IRB) chemically diverse compound library, the primary task was to select representative small molecules from various vendors. Given the diverse research and biological experiments conducted at the institute, the constructed library should cover a broad range of the chemical space. For instance, the selection included both fragments (having low molecular weight) and compounds known to act through covalent binding, among others. 

Of particular interest was the inclusion of compounds from the ChemDiv commercial library (\href{https://www.chemdiv.com/}{https://www.chemdiv.com/}), a leading provider of small molecules for drug discovery. First, we filtered the 1.5M compounds included in the ChemDiv catalog based on various criteria (e.g. discarding highly lipophilic compounds, see \hyperref[Navigation_Methods]{Methods}). Following the filtering process, 559k compounds remained, which was an order of magnitude higher than the intended size of the ChemDiv selection for the IRB library. In light of this, we implemented a clustering strategy, with the aim of preserving the chemical diversity of the overall set, rather than relying on a random selection of compounds. In brief, after characterizing all small molecules with chemistry-based CC signatures (see Fig \ref{Navigation_Fig1}a, b, d and e for A1, A2, A3 and A4 signatures’ applicability values, respectively), we clustered them using the k-means algorithm, setting the number of clusters to 35,000 (see \hyperref[Navigation_Methods]{Methods}). In fact, 328 of the clusters only included a single small molecule, while 51 of them contained 60 or more compounds (Fig \ref{Navigation_Fig1}c). To further validate our results, we assessed the standard deviations of signature features within each cluster, finding that the obtained values were consistently lower than those from random selections of compounds (Fig \ref{Navigation_Fig1}f). Indeed, the tSNE representation of all signatures revealed that compounds within the most populated clusters were closely placed together in the compressed 2D space, illustrating the coherence of the clustering outcomes (Fig \ref{Navigation_Fig1}g). Finally, for each cluster, we calculated the distances between the cluster centroid (an abstract 512-dimensional point, see \hyperref[Navigation_Methods]{Methods}) and the nearest compound, the farthest compound and the remaining compounds within the cluster, and a subset of compounds from other clusters. As expected, we observed that the distances between the centroid and the compounds within each cluster were consistently lower than those between the centroid and compounds from other clusters (Fig \ref{Navigation_Fig1}k, o). 

Finally, to select a balanced sample of each cluster, we selected the compounds that were closest and farthest to the cluster centroid as the cluster representatives. For clusters containing only a single compound, that compound was selected accordingly. All compounds selected as cluster representatives were subsequently chosen as ChemDiv representatives and thus included in the updated version of the IRB library, totalling 69,672 small molecules. Notably, the ChemDiv representatives exhibited physicochemical properties comparable to those from the entire ChemDiv library after filtering (Fig \ref{Navigation_Fig1}h-j and Fig \ref{Navigation_Fig1}l-n). 

%%%%%%%%%%%%%%%%
%%% FIGURE 1 %%%
%%%%%%%%%%%%%%%%

\begin{Figure_modified}
  \centering
  \includegraphics[width=1\linewidth]{figures/Navigation/Main/Diversity_random_v3.png}
  \caption{\textbf{Clustering a chemical library of compounds.}
    \textbf{a,b,d,e)} Distribution of CC signatures’ applicability values for A1, A2, A3 and A4, respectively. 
    \textbf{c)} Number of clusters (y-axis, log scale) having the specified number of molecules (x-axis). The expected number of molecules per cluster is 559k/35k \textasciitilde 16.
    \textbf{f)} For each cluster (x-axis, labeled from 0 to 34,999) standard deviations (y-axis) of the 128 features. Colored points represent the average standard deviation values of the features from signatures within the cluster, light gray bars show the range between the 20\textsuperscript{th} and 80\textsuperscript{th} percentiles of the distribution. Dark gray points represent the average standard deviation values of the features from randomly selected signatures outside the cluster.
    \textbf{g)} 2D tSNE representation of the 559k signaturized compounds (see \hyperref[Navigation_Methods]{Methods}). The top3 most populated clusters are colored accordingly, the legend indicates the number of small molecules within each of these clusters.
    \textbf{h,i,j,l,m,n)} Distributions of number of hydrogen bond acceptors, hydrogen bond donors, aromatic rings, molecular weight, topological polar surface area (tPSA) and alogP for the 559k ChemDiv compounds (after filtering, gray color) and the \textasciitilde70k selected representatives (light blue).
    \textbf{k,o)} Cosine and euclidean distances between the cluster centroid (an abstract 512-dimensional point, see \hyperref[Navigation_Methods]{Methods}) and the nearest compound, the farthest compound and the remaining compounds within each cluster, and a subset of compounds from other clusters.
}
  \vspace{-5mm}
  \rule[0ex]{\textwidth}{0.5pt}
  \vspace{-9mm}
  \label{Navigation_Fig1}
\end{Figure_modified}

\subsubsection{Qualitative visualization of the Chemical Space of small molecules}

To abstractly illustrate the chemical space of small molecules we subsampled compounds from Medina, Metabolights \cite{yurekten_metabolights_2024, haug_metabolights_2019}, CMAUP \cite{zeng_cmaup_2019, hou_cmaup_2024}, RepoHub \cite{corsello_drug_2017}, the Chemical Checker database \cite{duran-frigola_extending_2020} and the in-house chemically diverse IRB Library (herein named Chemical Diversity. Note that we here used the old version of the IRB Library, including 47k compounds). Medina’s natural product library represents one of the largest microbial extract libraries worldwide, providing a unique and invaluable resource for bioactive compound discovery (\href{https://www.medinadiscovery.com/natural-products-collection/}{https://www.medinadiscovery.com/natural-products-collection/}). On a similar note, Metabolights includes annotated metabolic compounds from various species, while CMAUP provides detailed information on active ingredients of useful medicinal plants. On the other hand, RepoHub (i.e. the Drug Repurposing Hub) is composed of existing clinical and marketed drugs, and is mainly aimed at repurposing known drugs for new therapeutic applications. Additionally, the Chemical Checker database comprises compounds with reported bioactivity in the public domain, and finally, the old version of the IRB proprietary library was specifically designed to maximize chemical diversity, covering a broad area of the synthetic chemical space.

To depict chemical differences between small molecule libraries, we generated chemical CC signatures for all molecules under study (see \hyperref[Navigation_Methods]{Methods}). Such signatures were in fact compact versions (128 dimensions) of state-of-the art chemical fingerprints, including ECFPs (CC A1 space), E3FPs (CC A2 space), together with embedded Murcko’s scaffold-based representations (CC A3 space), MACCs Keys (CC A4 space) and general physicochemical properties (CC A5 space). Additionally, we also generated bioactivity signatures, i.e. inferred representations of protein binding profiles (CC B4 space). 

Overall, we observed significant differences in the covered areas of the chemical space when using chemical signatures (A1-A5). For instance, there was very little overlap between compounds from the Chemical Diversity library and natural products from the Medina set (Fig \ref{Navigation_Fig2}). As expected, MetaboLights and CMAUP partially overlapped with Medina, while RepoHub and the Chemical Checker tended to extensively overlap with the synthetically diverse compounds from the Chemical Diversity (IRB) library. The use of bioactivity signatures (B4) led to the analogous conclusion: natural products and synthetic drugs exhibited distinct predicted protein binding profiles (Fig \ref{Navigation_Fig2}). Indeed, designed drugs and synthetically created compounds are imperfect human inventions with suboptimal properties and bioactivities, usually leading to limited efficacy and undesirable off-target effects. On the other hand, natural products and other ingredients derived from nature have evolved under the pressures of natural selection, resulting in tailored properties and refined selectivity profiles largely owing to their intricate chemical structures. Indeed, the scaffolds covered by synthetic compounds were far simpler than those included in natural products (Fig \ref{Navigation_FigS1}). 



%%%%%%%%%%%%%%%%
%%% FIGURE 2 %%%
%%%%%%%%%%%%%%%%

\begin{Figure_modified}
  \centering
  \includegraphics[width=1\linewidth]{figures/Navigation/Main/ALL_PRETTY.png}
  \caption{\textbf{Chemical space visualization of 6 distinct chemical libraries:} Medina, MetaboLights\cite{yurekten_metabolights_2024, haug_metabolights_2019}, CMAUP\cite{hou_cmaup_2024, zeng_cmaup_2019}, Chemical Diversity, RepoHub\cite{corsello_drug_2017} and the Chemical Checker\cite{duran-frigola_extending_2020}. For each combination of small molecule descriptor (A1-A5 and B4 CC Spaces) and compound library, tSNE 2D representation of the 31,052 generated signatures (see \hyperref[Navigation_Methods]{Methods}). Points are colored by library and 2D density.
}
  \vspace{-5mm}
  \rule[0ex]{\textwidth}{0.5pt}
  \vspace{-9mm}
  \label{Navigation_Fig2}
\end{Figure_modified}

\phantomsection
\subsubsection{The advent of bioactivity signatures}

The use of chemical small molecule descriptors is a common practice in most cheminformatics-related tasks, as we have seen when selecting a representative set of compounds to reduce the size of a large library and also when comprehensively visualizing the chemical space of small molecules. In the latter case, we derived analogous results using bioactivity signatures, which offer a promising framework to explore the chemical space of small molecules in a biologically relevant manner, potentially complementary to the purely chemical view. 

Many drug discovery efforts are rooted in the small molecule similarity principle, stating that structurally similar molecules do exhibit similar bioactivities when exposed to a biological system. Although the principle holds as a general guideline, the relationship between chemical structure and bioactivity is not straightforward and depends on multiple factors that are usually complex to trace. For instance, subtle modifications in a molecular structure may result in the disruption of the binding with its intended target protein or may alter key physicochemical properties, leading to suboptimal pharmacokinetics (e.g. poor absorption).

In the same way as similar molecules may exhibit very distinct bioactivities, non-similar molecules may present similar protein binding profiles. These cases are often overlooked by classical drug discovery efforts that purely rely on the comparison of chemical structures. This is where inferred bioactivity signatures become valuable, as they provide a means to capture functional similarities that are not evident from chemical analyses alone.

To illustrate the potential of bioactivity signatures, we focused our study on the chemical compounds reported to bind to the DNA damage-inducible transcript 3 mouse protein (Uniprot ID: P35639, Gene Name: DDIT3), as documented in the ChEMBL\cite{zdrazil_chembl_2024, gaulton_chembl_2017} database as of March 2024. In brief, we selected the reported active compounds (a total of 572) alongside 10k randomly selected molecules of the CC database, to serve as a background of the chemical space. The use of chemical signatures (CC A1 space, compressed representation of ECFPs) highlighted the diverse nature of these compounds. Remarkably, the binders spanned a wide area of the chemical space (Fig \ref{Navigation_Fig3}a), with cosine distances between them that were essentially indistinguishable from those observed against background compounds (Fig \ref{Navigation_Fig3}b). Although the considerable chemical differences exhibited by known binders, their characterization using bioactivity signatures (CC B4 space, inferred protein binding profiles) was notably more homogeneous. As shown in Fig \ref{Navigation_Fig3}c, most active compounds clustered within a well-defined and specific region of the chemical space due to significantly low distances between them (Fig \ref{Navigation_Fig3}d), emphasizing the high similarity between their bioactivity signatures. 


%%%%%%%%%%%%%%%%
%%% FIGURE 3 %%%
%%%%%%%%%%%%%%%%

\begin{Figure_modified}
  \centering
  \includegraphics[width=1\linewidth]{figures/Navigation/Main/Fig3_v3.png}
  \vspace{-10mm}
  \caption{\textbf{The advent of bioactivity signatures.}
    \textbf{a)} tSNE 2D representation of the 572 compounds (red, colored by density) reported to be active against P35639 (Uniprot ID, Gene Name: DDIT3) in the CC database (v.2024) derived from ChEMBL and BindingDB. Gray dots correspond to 10k randomly selected compounds that serve as a background of the chemical space. All compounds were previously characterized using the CC A1 signaturizer\cite{bertoni_bioactivity_2021}.
    \textbf{b)} Distribution of cosine distances between active compounds (red, 25k subsampled comparisons) and between active and inactive compounds (gray, 25k subsampled comparisons). All compounds were previously characterized using the CC A1 signaturizer\cite{bertoni_bioactivity_2021}.
    \textbf{c)} tSNE 2D representation of the 572 compounds (purple, colored by density) reported to be active against P35639 (Uniprot ID, Gene Name: DDIT3) in the CC database (v.2024) derived from ChEMBL and BindingDB. Gray dots correspond to 10k randomly selected compounds that serve as a background of the chemical space. All compounds were previously characterized using the CC B4 signaturizer\cite{bertoni_bioactivity_2021}.
    \textbf{d)} Distribution of cosine distances between active compounds (purple, 25k subsampled comparisons) and between active and inactive compounds (gray, 25k subsampled comparisons). All compounds were previously characterized using the CC B4 signaturizer\cite{bertoni_bioactivity_2021}.
    \textbf{e)} tSNE 2D representation of the 572 compounds (colored by chemical diversity, see \hyperref[Navigation_Methods]{Methods}) reported to be active against P35639 (Uniprot ID, Gene Name: DDIT3) in the CC database (v.2024) derived from ChEMBL and BindingDB. In short, the chemical diversity value represents the chemical redundancy from the neighboring environment of a bioactivity signature. Gray dots correspond to 10k randomly selected compounds that serve as a background of the chemical space. All compounds were previously characterized using the CC B4 signaturizer\cite{bertoni_bioactivity_2021}.
    \textbf{f)} A defined region of the chemical space (see previous subplot) is zoomed in to show (i) a compound having a high chemical diversity value (orange) as well as (ii) a pair of compounds having low chemical diversity (high chemical redundancy, blue). TS: Tanimoto Similarity. CD: Cosine Distance. 
}
  \vspace{-5mm}
  \rule[0ex]{\textwidth}{0.5pt}
  \vspace{-9mm}
  \label{Navigation_Fig3}
\end{Figure_modified}


To provide further insights into these results, we assigned a numerical value to each of the 572 active compounds representing the chemical diversity captured in their neighboring environment (see \hyperref[Navigation_Methods]{Methods}). In brief, the value illustrated how chemically similar a reference molecule was to those compounds having highly similar bioactivity signatures (top-10 NN). As shown in Fig \ref{Navigation_Fig3}e, active compounds had varying values of chemical diversity, ranging from low (blue, Tanimoto Similarities \textasciitilde 1) to high (orange, Tanimoto Similarities \textasciitilde 0.15) values. We then selected three representative compounds as exemplary cases to further illustrate these findings (Fig \ref{Navigation_Fig3}f). Compound \#1, with a diversity value of 0.81, exemplifies a significant chemical variation compared to its bioactivity-similar neighbors. In other words, compound \#1 was chemically different from its top-10 bioactivity-neighbors. On the other hand, compounds \#2 and \#3 represented the opposite case, having top-10 NNs with a maximum Tanimoto Similarity of 1 (leading to a chemical diversity value of 0). In fact, compound \#2 was the NN of compound \#3, and vice versa, pinpointing their nearly identical structures, differing only in the number of carbons in the aliphatic chain. 

Overall, these results highlight the impact that inferred bioactivity signatures may have in the context of drug discovery, offering an alternative framework to navigate the chemical space in an optimal manner for biological applications. In particular, a classical approach based on chemical features and descriptors, would have correctly anticipated the existing relationship between compounds \#2 and \#3 in terms of protein-ligand binding. However, chemical signatures alone would have failed to meaningfully relate compounds \#2 and \#3 with compound \#1. By leveraging bioactivity signatures, similarities between compounds \#1 and \#2 (or \#3) become significant (cosine distance of 0.06, see Fig \ref{Navigation_Fig3}d), and would have been anticipated even in the absence of chemical similarity (Tanimoto Similarity of 0.14 in both cases). 
