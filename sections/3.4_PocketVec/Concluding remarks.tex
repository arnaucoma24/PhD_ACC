\subsection{Concluding remarks}

We have presented PocketVec, a novel approach to generate vector-like protein pocket descriptors based on inverse docking and the chemogenomics principle that similar pockets bind similar ligands. A thorough assessment of its performance ranks it among the best available methodologies to characterize and compare protein druggable pockets, while overcoming some important limitations. We have also systematically searched for druggable pockets in the folded human proteome, using experimentally determined protein structures and AF2 models, identifying over 32,000 binding sites in more than 20,000 protein domains. We then derived PocketVec descriptors for each small molecule binding site and took advantage of their vector-like format to run an all-against-all pocket similarity search, exploring over 1.2 billion pairwise comparisons. Besides, we provide pre-computed descriptors for every identified human pocket together with the annotated Python code to generate new descriptors for any pocket of interest. We found that PocketVec descriptors are complementary to other, more classical, search strategies, enabling the identification of pocket similarities not revealed by structure- or sequence-based comparisons. As illustrative examples of applicability, we unveiled a clear relationship between pocket similarities, as defined by low PocketVec distances, and the probability of those pockets to bind the same compounds, as experimentally detected. Moreover, a systematic comparison of druggable pockets in protein kinases showed that kinase pairs with similar PocketVec descriptors also exhibited similar experimentally determined inhibition profiles. 

There have been recent attempts to identify druggable pockets at the proteome level\cite{wang_cavityspace_2022, konc_probis-fold_2022, sim_hproteome-bsite_2022, tsuchiya_possum_2023}. However, the novelty of our ligand-centric methodological approach, the accuracy of our descriptors and the systematic and exhaustive identification and characterization of binding pockets enabled the analyses of the effects of protein structural variation on pocket definition and small molecule binding at an unprecedented scale. Besides, the comprehensive list of human-derived pocket descriptors will become a valuable resource for the bio and cheminformatics communities.

This first generation of descriptors has been primarily designed for global analyses, such as the comprehensive characterization of all human druggable pockets. Indeed, our analyses have revealed dense clusters of similar pockets in distinct proteins for which no inhibitor has yet been co-crystalized, opening the door to strategies to prioritize the development of chemical probes to cover the druggable space\cite{carter_target_2019}. Moreover, our initial descriptors can be easily adapted to cater to specific tasks (i.e. exploring substrate specificity in a given protein family) by refining the selection of predefined lead-like molecules used or fine-tuning the similarity cutoff, thereby enhancing their performance. Of special interest are the anticipation of undesired off-targets as well as the guidance of rational polypharmacology, where single univalent molecules could be designed to target two proteins simultaneously, provided that their druggable pockets are similar enough\cite{duran-frigola_detecting_2017}. However, the main impact is likely to come from proteochemometric approaches, where a combination of ligand and target descriptors are used to train machine learning models\cite{fernandez-torras_connecting_2022}. It has been shown that structure-based descriptors of the targets are often superior to distinguish drug selectivity, although the sequence-based ones are often used when key protein structural details are lacking\cite{bongers_proteochemometrics_2019}. The generation of accurate descriptors derived for not yet described pockets in AF2 protein models overpasses this limitation, and opens up many possibilities. We envisage a scenario where small molecule and pocket descriptors combined are used to train AI-based generative models (e.g. to design new chemical entities that bind each protein druggable cavity\cite{jin_hierarchical_2020, blaschke_reinvent_2020}). Indeed, the estimated space of 10\textsuperscript{33} synthetically accessible drug-like molecules is mostly unexplored and represents a reservoir of potentially bioactive compounds\cite{polishchuk_estimation_2013}. Deep learning strategies have successfully designed new antibiotic scaffolds\cite{wong_discovery_2024} and placed 15 AI-designed drugs in clinical trials, including first-in-class molecules against several targets\cite{jayatunga_ai_2022}. Overall, accurate descriptors of druggable pockets might serve as a cornerstone for the development of generative AI approaches in drug discovery, offering unprecedented opportunities to expedite the design of a chemical toolbox to probe biology and, ultimately, to new therapeutics.