\subsection{Results and discussion}

%%%%%%%%%%%%%%%%%%%%%%%%%%%%%%%%%%%%%%%%%%%%%%%%%%%%%%%%%%%%%%%%
%%  PocketVec performance on the ProSPECCTs benchmark sets  %%%%
%%%%%%%%%%%%%%%%%%%%%%%%%%%%%%%%%%%%%%%%%%%%%%%%%%%%%%%%%%%%%%%%

\phantomsection
\subsubsection{PocketVec performance on the ProSPECCTs benchmark sets}
\label{PocketVec_ResultsAndDiscussion_PocketVec_performance_on_the_ProSPECCTs_benchmark}

The performance of PocketVec descriptors across ProSPECCTs datasets is assessed in terms of the AUROC and is shown in Fig \ref{PocketVec_Fig1}d. We observe that PocketVec descriptors are robust to varying definitions of the same pocket, as determined by different crystallized ligands (P1, AUROC 0.97). When restricting such definitions to chemically similar ligands, the performance is maximal (P1.2, AUROC 1.00). Similarly, our descriptors are robust against protein conformational changes, i.e. protein flexibility (P2, AUROC 0.96), and they are also able to distinguish identical pockets from those altered by 5 artificial mutations leading to changes in physicochemical and shape properties of pocket-lining residues. In these cases, we obtained more modest performances (P3--physicochemical changes and P4--both physicochemical and shape changes, AUROCs of 0.67 and 0.72, respectively). Reassuringly though, we observed a significant correlation between the number of artificial mutations (1 to 5) and the corresponding AUROC values (Pearson CC > 0.98, p-value < 0.005 in both P3 and P4, Fig \ref{PocketVec_FigS9}), which confirmed that an increasing number of mutations in the binding site came along with an improved ability to detect such differences using PocketVec descriptors. 

We also benchmarked our descriptors in more biologically relevant scenarios where, for instance, pockets binding similar small molecules are found in structurally different proteins. ProSPECCTs includes two datasets to address such cases: P5 includes pocket structures classified into 9 distinct ligand classes (e.g. HEM, ATP, NAD, etc.), and P7 includes a realistic set of binding site pairs reported to be similar in published literature, some of them identified in otherwise unrelated proteins. In these sets, PocketVec descriptors show performances with AUROCs of 0.64 and 0.87, respectively, demonstrating their ability to identify similar pockets in globally dissimilar proteins. It is important to note that two binding sites having identical (or chemically similar) crystallized ligands are not necessarily similar from a PocketVec perspective. Following the logic behind the similarity ensemble approach (SEA\cite{keiser_relating_2007}), in which targets are quantitatively compared based on the chemical similarity of the ensemble of their ligands, a pair of targets sharing a single active compound may not be significantly similar.

Finally, with the goal of defining a PocketVec distance threshold to classify any pocket pair of interest as either similar or dissimilar, we analyzed the behavior of the Matthew's Correlation Coefficient (MCC) at multiple cut-off values across the different ProSPECCTs datasets (Fig \ref{PocketVec_FigS10}). As expected, each definition of pocket similarity in the ProSPECCTs datasets suggested the choice of a different cut-off distance. For instance, the optimal cut-off in the mutation-related benchmarks (P3 and P4) was around 0.13, while in the most realistic set of binding site pairs reported to be similar in published literature was even lower, around 0.08. However, for general purposes and to minimize false negatives, we selected the distance threshold based on the P1 dataset, where similar pairs were predefined as identical pockets binding chemically distinct ligands whereas dissimilar pairs were unrelated pockets binding different ligands. In this way, we defined 0.17 as our PocketVec threshold distance, which was the value maximizing the MCC in ProSPECCTs P1. However, this threshold should not be regarded as an absolute standard, but rather as a general guideline for classifying a pocket pair of interest as either similar or dissimilar.

For the sake of completeness, we generated results for all combinations of docking methods (rigid--rDock, and flexible--SMINA) and reduced sets of chemical compounds (128 LLM and 128 fragments). The use of rDock and LLM (128) was still the best strategy according to the ProSPECCTs datasets (Fig \ref{PocketVec_FigS11}). Besides, we observed that the selection of compounds leading to the most variable results throughout the different pockets (i.e. high entropy) was, indeed, of great help to distinguish similar from dissimilar pockets in ProSPECCTs P1. This effect was even more pronounced in the fragments, where their lower complexity and molecular weight led to higher promiscuity and redundancy (Fig \ref{PocketVec_FigS12}).

Detailed plots for all ProSPECCTs datasets including ROC Curves, PR Curves and distributions of PocketVec distances, docking scores and pocket volumes are included in our \hl{GitLab repository} (see \hyperref[PocketVec_Code]{Code and data availability}). 



%%%%%%%%%%%%%%%%%%%%%%%%%%%%%%%%%%%%%%%%%%%%%%%%%%%%%%%%%%%%%%%%
%%%%%%%%%%  Comparison with existing strategies  %%%%%%%%%%%%%%%
%%%%%%%%%%%%%%%%%%%%%%%%%%%%%%%%%%%%%%%%%%%%%%%%%%%%%%%%%%%%%%%%


\subsubsection{Comparison with existing strategies}
\label{PocketVec_ResultsAndDiscussion_Comparison_with_existing_strategies}

Next, we compared the performance of PocketVec descriptors with state-of-the-art methodologies, as reported in several studies\cite{simonovsky_deeplytough_2020, scott_classification_2022, ehrt_benchmark_2018}, where the authors benchmarked many pocket comparison tools against the ProSPECCTs datasets, including six strategies based on pocket descriptors (Fig \ref{PocketVec_Fig1}d and Fig \ref{PocketVec_FigS11}). We used the AUROC as the standard performance measure to be able to compare our results with other available methodologies. However, for PocketVec descriptors, we also provide specific values of precision, accuracy, sensitivity, specificity, Matthew´s correlation coefficient and F1 score in each ProSPECCTs dataset (Fig \ref{PocketVec_FigS13}). Overall, PocketVec is the second-best ranked strategy in terms of the weighted average among datasets (0.89) and, indeed, it surpasses the median and the average performance in all ProSPECCTs datasets, apart from P3 (Table \ref{PocketVec_TableS1}). The top-scoring method is SiteAlign\cite{schalon_simple_2008}, which is alignment-dependent and based on the projection of residue descriptors into a triangle-discretized sphere that quantifies binding site similarity by minimizing distances between systematically generated cavity fingerprints obtained by moving one binding site with respect to the other. Thus, being specifically developed to compare pockets, it does not provide a unique descriptor for each binding site, which hampers the exploration of the pocket space in the same way molecular fingerprints do for the chemical space of small molecules. \hl{START} Other strategies are indeed alignment-free and provide unique representations for binding sites but, in addition to showing worse performances than PocketVec descriptors among ProSPECCTs datasets, also present several intrinsic limitations. For instance, KRIPO\cite{wood_pharmacophore_2012} and TIFP\cite{desaphy_encoding_2013} characterize protein-ligand binding interactions between receptors and bound ligands, which enables the identification of shared interaction patterns between pockets but limits their applicability domain to \textit{holo} structures. Additionally, FuzCav\cite{weill_alignment-free_2010} fingerprints count for specific pharmacophoric triplets of pocket-lining Cα, which enables the use of \textit{apo} structures but imply a simplistic representation of pockets. On a different note, Deeplytough\cite{simonovsky_deeplytough_2020} and BindSiteS-CNN\cite{scott_classification_2022} generate pocket descriptors by means of deep learning strategies, which makes them strongly dependent on training data and provide pocket embeddings that are difficult to interpret. \hl{END} Thus, overall, PocketVec represents a fast strategy to generate accurate pocket descriptors that overcome the aforementioned limitations, and it shows a higher performance at assessing pocket similarities than most current strategies (Fig \ref{PocketVec_FigS14}).
