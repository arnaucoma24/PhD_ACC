\subsection{Results and discussion}


\phantomsection
\subsubsection{PocketVec performance on the ProSPECCTs benchmark sets}

The performance of PocketVec descriptors across ProSPECCTs datasets is assessed in terms of the AUROC and is shown in Fig \ref{PocketVec_Fig1}d. We observe that PocketVec descriptors are robust to varying definitions of the same pocket, as determined by different crystallized ligands (P1, AUROC 0.97). When restricting such definitions to chemically similar ligands, the performance is maximal (P1.2, AUROC 1.00). Similarly, our descriptors are robust against protein conformational changes, i.e. protein flexibility (P2, AUROC 0.96), and they are also able to distinguish identical pockets from those altered by 5 artificial mutations leading to changes in physicochemical and shape properties of pocket-lining residues. In these cases, we obtained more modest performances (P3--physicochemical changes and P4--both physicochemical and shape changes, AUROCs of 0.67 and 0.72, respectively). Reassuringly though, we observed a significant correlation between the number of artificial mutations (1 to 5) and the corresponding AUROC values (Pearson CC > 0.98, p-value < 0.005 in both P3 and P4, Fig \ref{PocketVec_FigS9}), which confirmed that an increasing number of mutations in the binding site came along with an improved ability to detect such differences using PocketVec descriptors. 

We also benchmarked our descriptors in more biologically relevant scenarios where, for instance, pockets binding similar small molecules are found in structurally different proteins. ProSPECCTs includes two datasets to address such cases: P5 includes pocket structures classified into 9 distinct ligand classes (e.g. HEM, ATP, NAD, etc.), and P7 includes a realistic set of binding site pairs reported to be similar in published literature, some of them identified in otherwise unrelated proteins. In these sets, PocketVec descriptors show performances with AUROCs of 0.64 and 0.87, respectively, demonstrating their ability to identify similar pockets in globally dissimilar proteins. It is important to note that two binding sites having identical (or chemically similar) crystallized ligands are not necessarily similar from a PocketVec perspective. Following the logic behind the similarity ensemble approach (SEA\cite{keiser_relating_2007}), in which targets are quantitatively compared based on the chemical similarity of the ensemble of their ligands, a pair of targets sharing a single active compound may not be significantly similar.

Finally, with the goal of defining a PocketVec distance threshold to classify any pocket pair of interest as either similar or dissimilar, we analyzed the behavior of the Matthew's Correlation Coefficient (MCC) at multiple cut-off values across the different ProSPECCTs datasets (Fig \ref{PocketVec_FigS10}). As expected, each definition of pocket similarity in the ProSPECCTs datasets suggested the choice of a different cut-off distance. For instance, the optimal cut-off in the mutation-related benchmarks (P3 and P4) was around 0.13, while in the most realistic set of binding site pairs reported to be similar in published literature was even lower, around 0.08. However, for general purposes and to minimize false negatives, we selected the distance threshold based on the P1 dataset, where similar pairs were predefined as identical pockets binding chemically distinct ligands whereas dissimilar pairs were unrelated pockets binding different ligands. In this way, we defined 0.17 as our PocketVec threshold distance, which was the value maximizing the MCC in ProSPECCTs P1. However, this threshold should not be regarded as an absolute standard, but rather as a general guideline for classifying a pocket pair of interest as either similar or dissimilar.