\subsection{Concluding remarks}

Chemical signatures encode the physicochemical and structural properties of small molecules in the form of numerical descriptors, and they are at the core of chemical comparisons and search algorithms\cite{fernandez-torras_connecting_2022}. Recently, the widespread availability of bioactivity data has led to enhanced compound representations that capture their biological effects along with their chemical structures. However, unlike typical chemical descriptors that remain static, our bioactivity signatures are dynamic, evolving as new bioactivity data accumulates in databases or new strategies to process it appear.

We have now made available the complete computational protocol to modify or generate novel bioactivity spaces and signatures, describing the main steps needed to leverage diverse bioactivity data with the Chemical Checker using the predefined data curation pipeline. Moreover, we have illustrated the functioning of the protocol through four specific examples, including the incorporation of new compounds to an already existing bioactivity space, a change in the data pre-processing without altering the underlying experimental data, and the creation of two novel bioactivity spaces from scratch. 

Despite the regular updates, the expansion of the CC resource, beyond the original 25 spaces, is constrained by the availability and quality of public data sets. However, the systematic measurement of drug-induced perturbations in biological systems through omics technologies, together with advancements in phenotypic screening, are becoming increasingly common in the public and private sectors, providing an ever-growing corpus of biomolecular information. Besides, given the vast diversity of drug-like small molecules and the consolidation of generative AI as a novel strategy to design new chemical entities, computational tools are progressively becoming essential for initially estimating the biological effects of compounds. Overall, we envision that this protocol to create new bioactivity spaces and the associated descriptors will serve as a pivotal tool for exploring the bioactivity spectrum of compounds, effectively bridging the existing knowledge gap between the chemical structure and the exerted biological effects of small molecules.