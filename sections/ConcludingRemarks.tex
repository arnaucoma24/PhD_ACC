\chapter{Concluding remarks}
\label{concluding}
\clearpage

In the coming years, we anticipate a computational chemistry and biology landscape where drug candidates and biological entities will be primarily described by numerical vectors, leveraging available data from both public repositories and in-house experiments. These data would include structural features of the molecules and the targets, together with omics profiles, such as gene expression data, as well as large-scale biological networks and ontologies. Data would be linked at different levels with relatively simple operations, allowing for ultra-large, unbiased and systematic identification of the existing connections between the chemical space and the intricate biological space defined by disease biology. 

Indeed, the significant improvement of both chemical and protein descriptors has prompted the development of proteochemometric strategies, where machine learning models are trained on a combination of ligand and target representations \cite{bongers_proteochemometrics_2019}. These kinds of approaches have already shown superior performances in multi-target bioactivity prediction compared with classical methods \cite{torng_graph_2019}, although some results may be over-optimistic due to biases in the training datasets \cite{chen_hidden_2019}. Moreover, it has been shown that structure-based descriptors are often superior when a detailed definition of the target is needed (e.g. to distinguish drug selectivity among members of the same protein family), while sequence-based ones are better suited for more generic models, especially when key structural details are lacking \cite{bongers_proteochemometrics_2019}.

Only a subset of the approximate 20k canonical proteins that constitute the human proteome possess a binding site amenable to targeting (i.e. binding) by small molecules. Additionally, not all targetable proteins are necessarily disease-modifying, highlighting the need to carefully assess the pharmacological relevance of potential targets. In light of the limitations of small molecules in targeting certain proteins, alternative strategies have emerged to overcome them. For instance, peptides and antibodies often provide higher specificity and affinity than small molecules, particularly when targeting shallow or flat protein surfaces (e.g. protein-protein interaction surfaces). Other approaches involve proximity induced events, such as the use of molecular glues and PROTACs, in which the cellular machinery is hijacked to trigger protein degradation. On a different note, covalent inhibitors can target proteins, including intrinsically disordered proteins, by forming irreversible chemical bonds, leading to sustained inhibition. Together, these diverse strategies offer complementary avenues for expanding the scope and reach of drug discovery efforts. 

However, in view of the enormous complexity inherent to living organisms, as evidenced by the massive amount of bioactivity data gathered and released over the past decades, it is becoming increasingly necessary to move beyond the traditional “one disease-one target-one drug” rationale. Biological systems are highly interconnected, with multiple layers of organization and regulatory networks that contribute to their robustness and adaptability. A more holistic and global view is thus required to bridge translational gaps and accelerate drug discovery efforts to address complex and multifactorial human pathologies.

This thesis represents a very modest contribution in the ever-growing fields of cheminformatics and bioinformatics, encompassed within the wider and long-term process of drug discovery. We hope to add our tiny part and spur future computational efforts along these research lines.

