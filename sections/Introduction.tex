\chapter{Introduction}
\label{introduction}
\newpage


%%%%%%%%%%%%%%%%%%%%%%%%%%%%%%%%%%%%%%%%
%%%%%%%%%%%%%%%%%%%%%%%%%%%%%%%%%%%%%%%%
%%%%%%  NEW FIGURE ENVIRONMENT   %%%%%%%
%%%%%%%%%%%%%%%%%%%%%%%%%%%%%%%%%%%%%%%%
%%%%%%%%%%%%%%%%%%%%%%%%%%%%%%%%%%%%%%%%


\makeatletter
\newenvironment{figurehere}
{\def\@captype{figure}}
{}
\makeatother

\makeatletter
\newenvironment{Figure_modified}{%
\par\addvspace{12pt plus2pt}%
\def\@captype{figure}%
}{%
\par\addvspace{12pt plus2pt}%
}%

% Define a new Figure environment with top-of-page float placement
\newenvironment{Figure_modified2}{%
\par\addvspace{12pt plus2pt}%
\def\@captype{figure}%
\renewcommand{\@dblfloatplacement}{ht}%
\renewcommand{\@floatplacement}{ht}%
}{%
\par\addvspace{12pt plus2pt}%
}%

% Custom caption setup to ensure it integrates with the caption package and maintains the text size across all parts
\long\def\@makecaption#1#2{%
  \vskip\abovecaptionskip
  \sbox\@tempboxa{{\bfseries\sffamily\footnotesize #1}: \sffamily\footnotesize #2} % Apply bold, sans-serif and footnotesize to label, footnotesize to text
  \ifdim \wd\@tempboxa >\hsize
    {\bfseries\sffamily\footnotesize #1}: \sffamily\footnotesize #2\par % Maintain text size and styling if wrapping
  \else
    \global \@minipagefalse
    \hb@xt@\hsize{\hfil\box\@tempboxa\hfil}% Centering the caption if shorter than line width
  \fi
  \vskip\belowcaptionskip}
\makeatother


%%%%%%%%%%%%%%%%%%%%%%%%%%%%%%%%%%%%%%%%
%%%%%%%%%%%%%%%%%%%%%%%%%%%%%%%%%%%%%%%%
%%%%%%%%%%%%%%%%%%%%%%%%%%%%%%%%%%%%%%%%


%%% SIGNATURIZERS %%%

We systematically assess the relationship between stereoisomerism and bioactivity on a large scale, focusing on com poundtarget binding events, and use our findings to train novel deep learning models to generate stereochemically-aware bioactivity signatures for any compound of interest.

Moreover, the vector-like format of the resulting bioactivity descriptors enables them to be readily used in day-to-day cheminformatics tasks. For instance, we showed their utility to navigate the chemical space in a biologically relevant manner, unveiling shared mechanisms of action in the absence of chemical similarity. Additionally, we demonstrated that small molecule bioactivity descriptors provide a significant improvement in performance with respect to chemistry-restricted trained classifiers, across a series of biophysics and physiology activity prediction benchmarks. Indeed, our results showed that the added value of bioactivity descriptors increased together with the biological complexity of the classification tasks [7]