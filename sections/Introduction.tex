\chapter{Introduction}
\label{introduction}
\newpage


%%%%%%%%%%%%%%%%%%%%%%%%%%%%%%%%%%%%%%%%
%%%%%%%%%%%%%%%%%%%%%%%%%%%%%%%%%%%%%%%%
%%%%%%  NEW FIGURE ENVIRONMENT   %%%%%%%
%%%%%%%%%%%%%%%%%%%%%%%%%%%%%%%%%%%%%%%%
%%%%%%%%%%%%%%%%%%%%%%%%%%%%%%%%%%%%%%%%


\makeatletter
\newenvironment{figurehere}
{\def\@captype{figure}}
{}
\makeatother

\makeatletter
\newenvironment{Figure_modified}{%
\par\addvspace{12pt plus2pt}%
\def\@captype{figure}%
}{%
\par\addvspace{12pt plus2pt}%
}%

% Define a new Figure environment with top-of-page float placement
\newenvironment{Figure_modified2}{%
\par\addvspace{12pt plus2pt}%
\def\@captype{figure}%
\renewcommand{\@dblfloatplacement}{ht}%
\renewcommand{\@floatplacement}{ht}%
}{%
\par\addvspace{12pt plus2pt}%
}%

% Custom caption setup to ensure it integrates with the caption package and maintains the text size across all parts
\long\def\@makecaption#1#2{%
  \vskip\abovecaptionskip
  \sbox\@tempboxa{{\bfseries\sffamily\footnotesize #1}: \sffamily\footnotesize #2} % Apply bold, sans-serif and footnotesize to label, footnotesize to text
  \ifdim \wd\@tempboxa >\hsize
    {\bfseries\sffamily\footnotesize #1}: \sffamily\footnotesize #2\par % Maintain text size and styling if wrapping
  \else
    \global \@minipagefalse
    \hb@xt@\hsize{\hfil\box\@tempboxa\hfil}% Centering the caption if shorter than line width
  \fi
  \vskip\belowcaptionskip}
\makeatother


%%%%%%%%%%%%%%%%%%%%%%%%%%%%%%%%%%%%%%%%
%%%%%%%%%%%%%%%%%%%%%%%%%%%%%%%%%%%%%%%%
%%%%%%%%%%%%%%%%%%%%%%%%%%%%%%%%%%%%%%%%


\renewcommand{\thesubsection}{\thechapter.\arabic{subsection}}

\pdfbookmark[1]{Ghost Section}{ghost-section}

\subsection{A birds-eye view of Chemical Biology}
\label{Introduction_birds}

Every phenotypic trait in living organisms, shaped by the unceasing constraints of natural selection, ultimately traces back to underlying molecular events. However, as a result of millions of years of evolution, biological systems have become intrinsically complex, which often blurs the link between genotype and phenotype. Indeed, the human body comprises around 3·10\textsuperscript{13} cells, the fundamental unit of life \cite{hatton_human_2023, sender_revised_2016, bianconi_estimation_2013}. Depending on cell differentiation and environmental factors, the approximate 20k protein-coding genes encoded in the human genome are differently expressed to produce proteins, each varying in number from dozens to millions within each cell\cite{pertea_between_2010, beck_quantitative_2011, ezkurdia_multiple_2014, international_human_genome_sequencing_consortium_finishing_2004, ezkurdia_most_2015}. Proteins, subject to post-translational modifications (PTMs) and arranged in independent functional domains, play fundamental roles in biochemical processes, such as the catalysis of chemical reactions (e.g. kinases), the transduction of signals (e.g. insulin) or providing structural support in cells (e.g. actin)\cite{khoury_proteome-wide_2011, walsh_protein_2005}. At a lower scale, small molecules act as substrates, intermediates and products in many of those processes, such as those related to energy transfer (e.g. ATP) and signal transduction (e.g. serotonin). Within every cell, a plethora of reactions and interactions among biochemical entities, such as proteins and small molecules, occur under the realm of stochasticity driven by binding specificity. These events are intricately orchestrated and coordinated to sustain cell growth and survival. A disruption or malfunctioning in a single component of these processes may lead to the onset of human disease.

\subsection{Early small molecule pharmacology}
\label{Introduction_early}

Traditionally, the discovery of chemical compounds with the capacity to modulate or revert human disease states (i.e. drugs) was mostly serendipitous and entirely based on the therapeutic properties of substances natively present in nature (e.g. natural products) \cite{sneader_drug_2005}. This is the case of morphine \cite{sneader_drug_2005}, a powerful pain reliever alkaloid firstly isolated from opium in the early 19th century, and quinine \cite{achan_quinine_2011}, isolated from cinchona bark a couple of decades later, and used to treat malaria. Although the use of pure natural products further led to outstanding breakthroughs, such as Fleming’s revolutionary penicillin discovery in 1928 \cite{fleming_antibacterial_2001}, the idea that external compounds could be used to treat human diseases led to the development of the first synthetic small molecule drugs at the end of the 19th century (e.g. chloral hydrate and acetylsalicylic acid) \cite{sneader_drug_2005, vane_mechanism_2003}. At that time, a reductionist view of human disease prevailed, epitomized by the concept of the “magic bullet”, a term developed by the Nobel laureate and chemotherapy pioneer Paul Ehrlich in 1907 \cite{strebhardt_paul_2008, schwartz_paul_2004}. Ehrlich envisioned an optimal therapy in which an external agent (i.e. a chemical compound) selectively targeted and killed the disease-causing pathogen (e.g. bacteria) without any further effect in the host organism (i.e. the human body). Indeed, Ehrlich’s laboratory discovered arsphenamine in 1909, the first effective treatment for syphilis, also termed as the first designed magic bullet\cite{bosch_contributions_2008}. However, and unlike natural products, which are commonly characterized by intricate structures and tailored attributes shaped over eons of natural evolution\cite{grigalunas_chemical_2022}, synthetic drugs were (and still are) imperfect human inventions with suboptimal properties and bioactivities.

Despite pharmacological research being conducted from a primarily phenotypical perspective at that time, drug macromolecular counterparts simultaneously began to attract interest as a means to understand general biological processes and, in particular, the action of drugs. Paul Enrich and John N. Langley introduced the concept of receptors and laid the foundation for modern pharmacology at the beginning of the 20th century. Additionally, the concept of enzyme inhibition gained interest in the 1940s with the introduction of sulfonamides, a family of synthetic antibiotics that inhibited a bacterial enzyme essential for cell growth, thus underscoring proteins as relevant drug targets\cite{maehle_emergence_2002}.


The remarkable success of antibiotics in the mid-20th century, coupled with the post-WWII economic growth and advancements in organic chemistry and pharmacology, encouraged the pharmaceutical industry to explore novel therapeutic areas. Notably, the introduction of antipsychotics (e.g. chlorpromazine \cite{ban_fifty_nodate}), antihistamines (e.g. diphenhydramine \cite{simons_histamine_2011}), antidepressants (e.g. imipramine \cite{brown_clinical_2015}) and beta-blockers (e.g. propranolol \cite{srinivasan_propranolol_2019}) paved the way for significant advances in the treatment of psychiatric disorders, allergies and cardiovascular diseases. 




\renewcommand{\thesubsection}{\thechapter.\arabic{section}.\arabic{subsection}}
