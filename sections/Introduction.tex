\chapter{Introduction}
\label{introduction}
\newpage


%%%%%%%%%%%%%%%%%%%%%%%%%%%%%%%%%%%%%%%%
%%%%%%%%%%%%%%%%%%%%%%%%%%%%%%%%%%%%%%%%
%%%%%%  NEW FIGURE ENVIRONMENT   %%%%%%%
%%%%%%%%%%%%%%%%%%%%%%%%%%%%%%%%%%%%%%%%
%%%%%%%%%%%%%%%%%%%%%%%%%%%%%%%%%%%%%%%%


\makeatletter
\newenvironment{figurehere}
{\def\@captype{figure}}
{}
\makeatother

\makeatletter
\newenvironment{Figure_modified}{%
\par\addvspace{12pt plus2pt}%
\def\@captype{figure}%
}{%
\par\addvspace{12pt plus2pt}%
}%

% Define a new Figure environment with top-of-page float placement
\newenvironment{Figure_modified2}{%
\par\addvspace{12pt plus2pt}%
\def\@captype{figure}%
\renewcommand{\@dblfloatplacement}{ht}%
\renewcommand{\@floatplacement}{ht}%
}{%
\par\addvspace{12pt plus2pt}%
}%

% Custom caption setup to ensure it integrates with the caption package and maintains the text size across all parts
\long\def\@makecaption#1#2{%
  \vskip\abovecaptionskip
  \sbox\@tempboxa{{\bfseries\sffamily\footnotesize #1}: \sffamily\footnotesize #2} % Apply bold, sans-serif and footnotesize to label, footnotesize to text
  \ifdim \wd\@tempboxa >\hsize
    {\bfseries\sffamily\footnotesize #1}: \sffamily\footnotesize #2\par % Maintain text size and styling if wrapping
  \else
    \global \@minipagefalse
    \hb@xt@\hsize{\hfil\box\@tempboxa\hfil}% Centering the caption if shorter than line width
  \fi
  \vskip\belowcaptionskip}
\makeatother


%%%%%%%%%%%%%%%%%%%%%%%%%%%%%%%%%%%%%%%%
%%%%%%%%%%%%%%%%%%%%%%%%%%%%%%%%%%%%%%%%
%%%%%%%%%%%%%%%%%%%%%%%%%%%%%%%%%%%%%%%%


\renewcommand{\thesubsection}{\thechapter.\arabic{subsection}}

\pdfbookmark[1]{Ghost Section}{ghost-section}

\subsection{A birds'-eye view of Chemical Biology}
\label{Introduction_birds}

Every phenotypic trait in living organisms, shaped by the unceasing constraints of natural selection, ultimately traces back to underlying molecular events. However, as a result of millions of years of evolution, biological systems have become intrinsically complex, which often blurs the link between genotype and phenotype. Indeed, the human body is comprised by around 3·10\textsuperscript{13} cells, the fundamental unit of life \cite{hatton_human_2023, sender_revised_2016, bianconi_estimation_2013}. Depending on cell differentiation and environmental factors, the approximate 20k protein-coding genes encoded in the human genome are differently expressed to produce proteins, each varying in number from dozens to millions within each cell\cite{pertea_between_2010, beck_quantitative_2011, ezkurdia_multiple_2014, international_human_genome_sequencing_consortium_finishing_2004, ezkurdia_most_2015}. Proteins, subject to post-translational modifications (PTMs) and arranged in independent functional domains, play fundamental roles in biochemical processes, such as the catalysis of chemical reactions (e.g. kinases), the transduction of signals (e.g. insulin) or providing structural support in cells (e.g. actin)\cite{khoury_proteome-wide_2011, walsh_protein_2005}. At a lower scale, small molecules act as substrates, intermediates and products in many of those processes, such as those related to energy transfer (e.g. ATP) and signal transduction (e.g. serotonin). Within every cell, a plethora of reactions and interactions among biochemical entities, such as proteins and small molecules, occur under the realm of stochasticity driven by binding specificity. These events are intricately orchestrated and coordinated to sustain cell growth and survival. A disruption or malfunctioning in a single component of these processes may lead to the onset of human disease.

\subsection{Early small molecule pharmacology}
\label{Introduction_early}

Traditionally, the discovery of chemical compounds with the capacity to modulate or revert human disease states (i.e. drugs) was mostly serendipitous and entirely based on the therapeutic properties of substances natively present in nature (e.g. natural products) \cite{sneader_drug_2005}. This is the case of morphine \cite{sneader_drug_2005}, a powerful pain reliever alkaloid firstly isolated from opium in the early 19th century, and quinine \cite{achan_quinine_2011}, isolated from cinchona bark a couple of decades later, and used to treat malaria. Although the use of pure natural products further led to outstanding breakthroughs, such as Fleming’s revolutionary penicillin discovery in 1928 \cite{fleming_antibacterial_2001}, the idea that external compounds could be used to treat human diseases led to the development of the first synthetic small molecule drugs at the end of the 19th century (e.g. chloral hydrate and acetylsalicylic acid) \cite{sneader_drug_2005, vane_mechanism_2003}. At that time, a reductionist view of human disease prevailed, epitomized by the concept of the “magic bullet”, a term developed by the Nobel laureate and chemotherapy pioneer Paul Ehrlich in 1907 \cite{strebhardt_paul_2008, schwartz_paul_2004}. Ehrlich envisioned an optimal therapy in which an external agent (i.e. a chemical compound) selectively targeted and killed the disease-causing pathogen (e.g. bacteria) without any further effect in the host organism (i.e. the human body). Indeed, Ehrlich’s laboratory discovered arsphenamine in 1909, the first effective treatment for syphilis, also termed as the first designed magic bullet\cite{bosch_contributions_2008}. However, and unlike natural products, which are commonly characterized by intricate structures and tailored attributes shaped over eons of natural evolution\cite{grigalunas_chemical_2022}, synthetic drugs were (and still are) imperfect human inventions with suboptimal properties and bioactivities.

Despite pharmacological research being conducted from a primarily phenotypical perspective at that time, drug macromolecular counterparts simultaneously began to attract interest as a means to understand general biological processes and, in particular, the action of drugs. Paul Enrich and John N. Langley introduced the concept of receptors and laid the foundation for modern pharmacology at the beginning of the 20th century. Additionally, the concept of enzyme inhibition gained interest in the 1940s with the introduction of sulfonamides, a family of synthetic antibiotics that inhibited a bacterial enzyme essential for cell growth, thus underscoring proteins as relevant drug targets\cite{maehle_emergence_2002}.


The remarkable success of antibiotics in the mid-20th century, coupled with the post-WWII economic growth and advancements in organic chemistry and pharmacology, encouraged the pharmaceutical industry to explore novel therapeutic areas. Notably, the introduction of antipsychotics (e.g. chlorpromazine \cite{ban_fifty_2007}), antihistamines (e.g. diphenhydramine \cite{simons_histamine_2011}), antidepressants (e.g. imipramine \cite{brown_clinical_2015}) and beta-blockers (e.g. propranolol \cite{srinivasan_propranolol_2019}) paved the way for significant advances in the treatment of psychiatric disorders, allergies and cardiovascular diseases. Although numerous drugs were still discovered either serendipitously (e.g. being primarily designed to treat other pathologies) or empirically (e.g. testing large collections of compounds for desired biological activity without necessarily understanding the underlying molecular mechanisms), the first examples of designed drugs meant to target specific proteins began to emerge\cite{drews_drug_2000}.

Indeed, the increasing interest in proteins as potential drug targets changed the way in which researchers conceived drug discovery. Living organisms were progressively no longer seen as black boxes but as systems with specific, targetable biochemical entities (e.g. proteins) whose modulation could result in therapeutic effects, laying the foundation for the rational design of new drugs (aka reverse pharmacology). 

\subsection{Rational drug design}
\label{Introduction_rational}

The main hypothesis behind rational drug design is that by modulating the activity of a specific target associated with a disease (e.g. a protein), it is possible to halt or revert the progression of the disease or to alleviate its symptoms. The modulation is usually achieved with an external compound (i.e. a drug) that interacts with the target to exert its therapeutic action, ideally without causing significant side effects.

Of special interest is propranolol, the first clinically successful beta-blocker, developed by Nobel laureate Sir James Black in the early 1960s. It is primarily used to treat angina and hypertension\cite{srinivasan_propranolol_2019}. The intended targets of propranolol were the beta-adrenergic receptors, under the hypothesis that blocking the action of adrenaline or noradrenaline in those receptors would decrease blood pressure and heart rate. Indeed, Black based his design on pronethalol, an already known antagonist of the beta-adrenergic receptors, and, by slightly modifying its chemical structure, he managed to increase the potency of the compound and to reduce its carcinogenicity\cite{black_new_1964}. In this way, propranolol was designed with a blended empirical-rational drug design approach.

The 1960s also marked the advent of protein crystallography, leading to a paradigmatic shift in the development of new drugs. Myoglobin and hemoglobin were the first protein structures to be experimentally resolved using X-ray crystallography, a breakthrough for which Kendrew and Perutz were awarded the Nobel Prize in Chemistry back in 1962\cite{kendrew_three-dimensional_1958, muirhead_structure_1963}. Indeed, the availability of protein structures opened the door not only to understanding protein function from a molecular perspective, but also to studying fine binding details between endogenous small molecules (e.g. heme\cite{kendrew_three-dimensional_1958, muirhead_structure_1963}) or drugs (e.g. methotrexate \cite{bolin_crystal_1982}) and their respective targets. This shift provided a comprehensive molecular framework to disrupt or enhance the effects of endogenous ligands as well as to improve the potency of known drugs. Interestingly, the 1990s saw the approval of the first drugs whose design was based on specific structural details of the target protein binding sites, such as saquinavir (targeting the HIV protease to treat HIV infections) and dorzolamide (targeting the carbonic anhydrase to treat eye hypertension)\cite{rondeau_protein_2008, leelananda_computational_2016}. 

Apart from exhibiting sufficient binding affinity towards the intended target, drugs need to be optimized for metabolic stability and minimal toxicity. In addition, and depending on the route of administration, drugs must first be efficiently absorbed into the bloodstream, which requires a delicate balance between aqueous solubility and biological membrane permeability, factors that critically determine its bioavailability. Once in the bloodstream, drugs will diffuse throughout the body to eventually reach its target tissue(s), potentially crossing specific biological barriers (e.g. blood-brain barrier) and ultimately traversing cell membranes when necessary. Furthermore, drugs must be efficiently eliminated from the body to prevent its accumulation and subsequent toxicity. Overall, rational drug design is a multi optimization process in which multiple variables need to be considered simultaneously, typically requiring an iterative cycle involving computational modeling, chemical synthesis, biological testing and, eventually, clinical evaluation\cite{waring_analysis_2015, patrick_introduction_2023}. 

In the context of rational drug design, the “magic bullet” concept introduced by Ehrlich in the early 20th century, which envisioned the specific target of disease-causing agents, evolved into the more modern but still reductionist “one disease-one gene-one drug” paradigm. Although such an enticing view has led to numerous successful drug approvals, it has inherent limitations \cite{samsdodd_target-based_2005, swinney_how_2011}. Indeed, not all genes (i.e. proteins) are amenable for modulation by a drug-like compound, such as those lacking a defined binding site in their three-dimensional constituting domains. Additionally, the massive release of biological data (i.e. omics) following the sequencing of the human genome in the early 2000s \cite{venter_sequence_2001, international_human_genome_sequencing_consortium_initial_2001, field_omics_2009, mardis_decades_2011} has uncovered an overwhelming complexity of biological systems\cite{kitano_systems_2002}, showing that human disease cannot be fully understood by studying their individual components alone. In fact, complex diseases (e.g. cancer, diabetes) usually result from multiple disruptions within intricate biological networks (e.g. protein-protein interactions), influenced by both genetic and environmental factors, that cannot be pinned down to a single gene malfunction. To address this, emerging fields such as systems pharmacology aim at adopting a more holistic view on biological systems, thus bridging the gap between phenotypic and target-based drug discovery.

In any case, whether drug targets are single proteins or complex biological processes, most approved drugs are still new molecular entities\cite{mullard_2023_2024}. Indeed, small molecules are an excellent tool to probe biological functions and the primary choice of pharmaceutical companies, as they are easy to manufacture, store, and distribute, and synthetic chemists can conceive a broad variety of them.

\subsection{The Chemical Space of small molecules}
\label{Introduction_chemicalspace}

Out of the essentially infinite distinct organic molecules one can dream of, only a portion have suitable properties to be used as drugs and have thus potential for human clinical use. In a prospective work conducted in the late 1990s, Lipinksi and colleagues noted that a significant proportion of small molecules that had entered Phase II clinical trials at that time (2,245 compounds) shared chemical and physical properties that tended to fall within specific ranges (e.g. molecular weight <500 Da)\cite{lipinski_experimental_2001}. This observation led to the formulation of the Lipinski's “Rule of Five”, a set of guidelines to prioritize drug-like compounds in early stages of drug discovery, further enabling the exploration of the chemical space (i.e. the infinite set of organic molecules) in a manner relevant to drug development\cite{lipinski_navigating_2004, dobson_chemical_2004, reymond_chemical_2010}. Some studies estimate the number of drug-like small molecules to be in the range of 10\textsuperscript{33}-10\textsuperscript{60}\cite{bohacek_art_1996, polishchuk_estimation_2013}. However, and despite the huge amount of potentially bioactive compounds that exist within the chemical space, the number of approved small molecule drugs is comparatively very small, standing at 2,781 as of March 2024 (DrugBank v 5.1.12). This stark contrast underscores the significant cost and complexity associated with drug development efforts, which on average take billions of US\$ and more than a decade to bring a new drug to the market\cite{wouters_estimated_2020, sertkaya_costs_2024, dimasi_innovation_2016, hinkson_accelerating_2020, dimasi_research_2020}, primarily due to high attrition rates in clinical trials\cite{sun_why_2022}. 

Before biological tests are performed to evaluate the therapeutic potential of a candidate compound, it first has to be synthesized in sufficient amounts. The size of commercial (e.g. Enamine REAL Space) and proprietary libraries (e.g. Merck MASSIV 2018) has dramatically increased in the last decade, including up to 10\textsuperscript{9} and 10\textsuperscript{20} accessible chemical compounds, respectively, with synthetic routes success rates that may exceed 80\%\cite{stein_virtual_2020, hoffmann_next_2019, klingler_sar_2019}. This is mainly due to the implementation of strategies based on two- or three-step three-component reaction sequences and the availability of starting materials with pre-validated chemical reactivity \cite{grygorenko_generating_2020}. In addition, high-throughput screening (HTS) assays have penetrated the public research sector in the last years, providing depth of biological annotation to compound collections\cite{subramanian_next_2017, corsello_discovering_2020}. This is reflected in the increasing number of bioactive small molecules cataloged in open databases, which already amount to over two million entries\cite{gaulton_chembl_2017, wang_pubchem_2017}. Additionally, the effects of small molecules in biological systems are multifaceted and often annotated through various levels of biological complexity. For instance, a compound may bind specific biochemical targets, disrupt or modulate certain signaling pathways, rewire protein interaction networks, induce changes in gene expression, modify protein abundance or alter cell morphology.

As the number of available compounds and associated biological data continues to grow, the need for effective methods to manage, analyze, and interpret this vast amount of information has become increasingly important. This demand has spurred the development of fields such as cheminformatics and bioinformatics, which aim to increase the efficiency of drug discovery processes. 


\subsection{Cheminformatics and the similarity principle}
\label{Introduction_cheminformatics}

Blending chemical knowledge, biological data and computer science in the context of drug discovery has been a cornerstone for the cheminformatics field since the advent of computation \cite{brown_2018, engel_2006}. Small molecules are usually depicted in the form of SMILES, a string-based molecular representation that, although ambiguous (i.e. a single molecule may be described with multiple SMILES), is very intuitive and has been broadly adopted in the field\cite{weing}.




\renewcommand{\thesubsection}{\thechapter.\arabic{section}.\arabic{subsection}}
