\chapter{Introduction}
\label{introduction}
\newpage

\makeatletter
\newenvironment{figurehere}
{\def\@captype{figure}}
{}
\makeatother

\makeatletter
%Figure environment without an float format
\newenvironment{Figure_modified}{%
\par\addvspace{12pt plus2pt}%
\def\@captype{figure}%
}{%
\par\addvspace{12pt plus2pt}%
}%

% Custom caption setup to ensure it integrates with the caption package and maintains the text size across all parts
\long\def\@makecaption#1#2{%
  \vskip\abovecaptionskip
  \sbox\@tempboxa{{\bfseries\sffamily\footnotesize #1}: \sffamily\footnotesize #2} % Apply bold, sans-serif and footnotesize to label, footnotesize to text
  \ifdim \wd\@tempboxa >\hsize
    {\bfseries\sffamily\footnotesize #1}: \sffamily\footnotesize #2\par % Maintain text size and styling if wrapping
  \else
    \global \@minipagefalse
    \hb@xt@\hsize{\hfil\box\@tempboxa\hfil}% Centering the caption if shorter than line width
  \fi
  \vskip\belowcaptionskip}
\makeatother

\begin{Figure_modified}
 \begin{center}
  \includegraphics[width=\textwidth]{example-image}
  \caption{Correlation between inhibition profiles and PocketVec descriptors. All the analyses have been performed on the data obtained from Kleager et al. (left panels).
    \textbf{a, g)} Inhibition matrix between protein kinases, and small molecule kinase inhibitors binarized at 30 nM. Both kinases and inhibitors are sorted by the number of active inhibition events. Orange dots indicate inhibition and white dots indicate no inhibition.
    \textbf{b, h)} Distributions of PocketVec distances grouped by the number of shared inhibitors between kinases (0, 1-3 and 4 or more). The number of kinase pairs per number of shared inhibitors is specified in parenthesis.
    \textbf{c, i)} Enrichments (Fisher’s exact test) in similar inhibition profiles (Jaccard Similarity >0.5) for those kinase pairs being similar in terms of PocketVec distance (red: <0.17, orange: <0.10), structural similarity (TM-score >0.85) and sequence identity (>35\%). For comparison, the results obtained with randomly selected kinase pairs (gray) are also included. Circle areas are proportional to the corresponding ORs and p-values are specified in the center with the following format: * p-value < 0.05, ** p-value < 0.001.
    \textbf{d, j)} Pairwise kinase comparisons. Rows and columns correspond to alphabetically sorted kinases (by Uniprot ID). Upper triangular matrices: kinases are compared on the basis of their experimentally determined inhibition profiles. Each square represents the Jaccard similarity between the inhibition profiles of two targets: the higher the Jaccard similarity, the more similar the corresponding inhibition profiles. Lower triangular matrices: kinases are compared on the basis of their PocketVec descriptors (employing the minimum distance among all PocketVec descriptors in the Protein Kinase Domains). The color of each square indicates the minimum PocketVec distance between two targets: the lower the PocketVec distance (red), the more similar the kinases are at pocket level according to our descriptors.
    \textbf{e, k)} Relationship between structural similarity (x-axis, Max. TM-score) and PocketVec distances (y-axis) between pairs of protein kinases. Each point represents a kinase pair and is colored and sized in terms of the similarity between experimentally determined inhibition profiles.
    \textbf{f, l)} Relationship between sequence similarity (x-axis, Max. Seq. Id.) and PocketVec distances (y-axis) between pairs of protein kinases. Each point represents a kinase pair and is colored and sized in terms of the similarity between experimentally determined inhibition profiles.
    \label{Intro1}}
 \end{center}
\end{Figure_modified}

