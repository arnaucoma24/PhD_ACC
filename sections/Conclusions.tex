\chapter{Concluding remarks}
\label{concluding_remarks}




Other modalities in drug discovery: peptides, antibodies, molecular glues, ProTACCs, etc



Limitations (maybe in the concluding remarks?)

Limitations of rational drug design -alternative approaches. ProTACCS, monoclonal antibodies, etc
Limitations of stereoisomers -0 is not necessarily inactive
Why we did no use chiral ecfp? Take discussion from the rebuttal

Check concluding remarks Stereoisomerism\_Draft\_v2


Limitations of PocketVec

Limitations alphafold (protein space, point mutations, etc)



Difficult targets (IDP, PPIs -shallow surfaces -drugs don't have enough affinity)
Conformational modifications (PTM induce active/inactive protein conformations).
Other modalities (ProTACCS)
AF3?

Pockets -types of pockets? Paper PLOS ONE 



COCB:

The significant improvement of both chemical and protein descriptors has prompted the development of proteochemometric strategies, where machine learning models are trained on a combination of ligand and target representations [59*]. Indeed, these kinds of approaches have already shown superior performances in multi-target bioactivity prediction compared with classical methods [60], although some results may be over-optimistic due to bias in the training datasets as pointed out in the study by Chen et al. [61]. Moreover, Bongers et al. [59*]showedthat structure-based descriptors are often superior when a detailed definition of the target is needed (i.e. to distinguish drug selectivity among members of the same protein family), while sequence-based ones are better suited for more generic models, especially when key structural details are lacking.



Number of proteins is 20**300 [REF]... only few thousand chosen by evolution, but protein design is a … field -novel proteins with fine-tuned functions. 

REF: Dobson, Christopher M. "Chemical space and biology." Nature 432.7019 (2004).

