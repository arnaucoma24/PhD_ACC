\subsection{Introduction}


Small molecules are a great tool to probe biology and, still, the main asset of pharmaceutical companies. The last years have seen a surge of ever more complex biological high-throughput assays involving the use of chemical compounds, and databases committed to gathering bioactivity data associated to small molecules are expanding \cite{zdrazil_chembl_2024, kim_pubchem_2023}. Moreover, the widespread availability of computational resources\cite{tetko_bigchem_2016} and artificial intelligence techniques has been pivotal to leverage such amounts of data\cite{von_lilienfeld_retrospective_2020}. 

From the computational perspective, small molecules are typically characterized by numerical descriptors encoding physicochemical or topological features\cite{fernandez-torras_connecting_2022}. Compounds can be further described using their biological activities (e.g. the targets they interact with), which represents a complementary strategy that extends the small molecule similarity principle beyond conventional chemical properties\cite{duran-frigola_extending_2020}. Unfortunately, experimental bioactivity data are sparse and only available for a limited set of well-characterized compounds. To overcome these coverage issues, we recently trained a collection of deep neural networks able to infer bioactivity signatures for any compound of interest (i.e. Signaturizers), even when little or no experimental information is available for them\cite{bertoni_bioactivity_2021}. The Signaturizers are able to infer 25 different bioactivity types, from target profiles to cellular responses or clinical outcomes. However, the original Signaturizers are built on 2D representations of molecules and are thus not able to capture subtle, but often meaningful, bioactivity differences between stereoisomers. Indeed, stereochemistry and chirality play pivotal roles in pharmacology\cite{scott_stereochemical_2022, h_brooks_significance_2011}, often driving supramolecular recognition processes crucial in drug design. Biological matter is intrinsically chiral\cite{inaki_cell_2016} (e.g. amino acids) and stereoisomeric small molecule drugs may exhibit different therapeutic and toxicological effects\cite{mcconathy_stereochemistry_2003, smith_chiral_2009}. For example, the antidepressant Citalopram is administered as a mixture of two enantiomers (i.e. racemate), although only one of them is pharmacologically active\cite{snchez_escitalopram_2004, sanchez_pharmacology_2006}. However, in some other cases, one of the enantiomers is associated with toxic side effects. This is the case of the antiarthritic drug Penicillamine, administered as an enantiomerically pure compound ((S)-Penicillamine) since (R)-Penicillamine acts as a pyridoxine (vitamin B6) antagonist and is thus toxic\cite{smith_chiral_2009, williams_enantiomers_1990}. We now present novel deep learning models to generate stereochemically-aware bioactivity signatures for any compound of interest, which we call Signaturizers3D, that overcome the inherent limitations of our original Signaturizers. 

